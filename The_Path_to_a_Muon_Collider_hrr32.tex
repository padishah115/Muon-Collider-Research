\documentclass{article}
\usepackage{graphicx} % Required for inserting images
\usepackage{multicol}

\title{The Path to a Muon Collider}
\author{Supervised by Dr Melissa Uchida \\ Department of Physics, University of Cambridge}
\date{May 2023}

\begin{document}
	
	\maketitle
	
	
	\tableofcontents
	
	\section{Abstract}
	
	\par Muon colliders are an as-of-yet unrealised technology. However, this does not diminish their potential as a viable candidate for the next generation of high-energy colliders. The theoretical centre-of-mass energies for proposed muon collisions are high enough to probe the Higgs mechanism, with the possibility of producing multiple Higgs events and probing any self-coupling behaviours of the Higgs field. Muon colliders offer significant advantages over their hadron counterparts- a muon collider would take up significantly less space than a hadron collider producing similarly energetic collisions. Thus, focusing research and development on muon colliders could drastically reduce the size and cost of future colliders. Muons suffer from less radiative energy losses than electrons due to their larger mass, making them more suitable for lepton-antilepton collisions. Current challenges inhibiting the development of muon colliders include the large emittance of muon beams produced by conventional methods (currently being combated by ionization cooling methods such as those at MICE) and the short rest-frame lifetime of muons (2.2$\mu$s).
	
	
	\section{History and Context}
	
	
	\par Particle colliders are an immensely valuable technology in furthering our understanding of the Standard Model. For example, the work of the ATLAS and CMS experiments led to the observation of the Higgs Boson in 2012[1] at the Large Hadron Collider (LHC) in high-energy collisions. Previously, from 1989 until its deconstruction, the Large Electron-Positron Collider (LEP) was used to probe energies of around 91 GeV[2], corresponding to the mass of the Z boson, enabling the deeper investigation of its decay and therefore cementing our understanding of the Standard Model.
	\par Even earlier, the Intersecting Storage Rings (ISR) collided two proton beams in January 1971[3], making the ISR the first hadron collider in history, using beams with up to 26 GeV in energy[4].
	\par Additionally, work by the teams at the Stanford Positron Electron Asymmetric Rings (SPEAR) collider at the SLAC National Laboratory in California led to the discovery of the J/$\psi$ meson in 1974 (simultaneously with the team at Brookhaven National Laboratory in Long Island), as well as the discovery of the tau lepton[5], leading to Martin Perl receiving a Nobel Prize in 1995. These represent two leaps in our understanding of the Standard Model, the former allowing us to investigate the charmonium quark, and the latter confirming the existence of three generations or “families” in the Standard Model.
	\par This wealth of 20th century advances in high-energy physics motivates the construction of another generation of particle colliders, with the proposed muon collider emerging as a strong contender for several reasons. Namely, its use of fundamental particles (muons and antimuons) rather than hadrons as in the LHC and proposed future hadron colliders like the Future Circular Collider (FCC). Such colliders are limited by the fact that hadrons are bound states of quarks and gluons, and are thus not point-like. As such, they cannot impart all of their energy into a single collision.
	
	
	\section{Proposed Design}
	
	\par Much like previous colliders, a proposed future muon collider would be a circular collider, approximately 10km in diameter[6], significantly smaller than proposed designs for the FCC, and would probe significantly higher energies than its hadron collider competitors. A muon collider operating at 14 TeV would allow physicists to probe energy levels similar to those probed by a 100 TeV FCC[7]. Muon colliders would require high-gradient RF cavity resonators in order to accelerate muons to high enough speeds to reach the point of collision before decay.
	\par Muon colliders benefit from less synchrotron radiation than electron-positron competitors, eliminating the need for a linear accelerator[8].
	\par The figure below is a schematic representation of a proposed collider design, consisting of a muon injector, accelerator ring and a collider.
	
	\begin{center}
		\includegraphics[scale=0.5]{muon_collider.png}
		\par \textit{Schematic representation of a potential muon collider[6].}
	\end{center}
	
	
	
	
	\par The injector would consist of a solenoid, responsible for the capture of pions produced by the proton beam as well subsequent bunching of the muon beams, a muon cooling channel such as that proposed by MICE (see section 5) which would need to cool the beam by approximately five orders of magnitude[6], and an accelerating unit. All of this must take place long before the muon decay lifetime (2.2$\mu$s) elapses.
	
	\section{The Physics of a Muon Collider}
	
	\subsection{Theoretical Advantages}
	
	
	\par Muons are fundamental particles, and thus the entirety of their energy is immediately available to reactions in a collider, as opposed to hadrons, which are composed of quarks and gluons. Whereas hadron collisions take place between hadron constituents with only a fraction of the total hadron energy, muons contain no such constituents and impart the entirety of their energy to the collision[7]. The energy scales probed by a muon beam collision can be 7 times greater than scales probed by hadron beams of a similar energy[6]. Therefore, muons present us with the opportunity to create colliders with much greater efficiency than their competitors, such as the Future Circular Collider (FCC). 
	
	\begin{center}
		\includegraphics[scale=0.35]{mu_vs_p.png}
		\par \textit{Taken from [6]. The diagram at which energy values muon collision cross section equals proton collision cross-section. Muon collisions achieve the same cross section for significantly less energy due to being fundamental particles. The dashed line assumes that the processes are similar for muon and proton collisions, whereas the solid line corrects for interactions explained by QCD in a proton-proton collider absent from muon collisions.}
	\end{center}	
	
	
	\par Additionally, muons have a mass about $206$ times greater than electrons, reducing their radiative energy losses by a factor of about $10^{9}$ while undergoing centripetal acceleration in a collider ring when subjected to a magnetic field[7][9]. This is a tremendous theoretical victory in reducing energy losses during the operation of a particle collider, allowing a theoretical muon collider to accelerate muon beams through multiple passes of the collider ring. 
	
	$$\textrm{radiated energy/orbit} = \frac{E^4}{m^4r}$$
	
	\par Electron-positron beams, which must be accelerated in a linear collider before collision inside a single detector are thus at a disadvantage to muon beams, which can be accelerated several times in “multi-pass” rings without significant reduction of beam luminosity before being passed into a collider ring for (potentially) several collisions. This increases the number of detectors that can be fitted to a muon collider as opposed to an electron-positron collider[8].
	\par The energy scales offered by muon-antimuon collisions are suitable for probing the self-coupling of the Higgs field (producing double- or triple-Higgs field events)[6], investigating the possible substructures of quarks, or searching for supersymmetric particles in currently inaccessible regions of the energy scale. The promise of investigating such energy levels, as well as the fact that muon colliders are theoretically more energy-efficient than their electron and hadron counterparts, may foster a wider interest in muon colliders in the near future.
	
	
	\subsection{Theoretical Disadvantages}
	
	\par Muon colliders, despite their promise, have not yet been constructed for a number of reasons. One of the greatest obstacles to the construction of a successful muon collider is the incredibly short length of the muon lifetime in its rest frame (2.2$\mu$s). The Muon Ionization Cooling Experiment (MICE) combats this challenge by using a repeating lattice of absorber material[10], allowing marked decrease of muon beam emittance over timescales shorter than the muon’s lifetime. However, the ionization cooling process further complicates the matter of the short muon lifetime by reducing the beam’s longitudinal momentum, reducing the distance muons can travel before decay. This problem can be mitigated by the use of radiofrequency (RF) cavities[11], which can accelerate muons to the large velocities necessary for a muon to travel around a collider ring before decay. Experiments conducted by Fermilab in the US have simulated the use of RF cavities with gradients of 30MV/m at 800MHz in order to prove their viability in accelerating muons to high velocities[12]. The design and construction of these RF cavities is an ongoing project, and they as of yet have not been used to accelerate muon beams. High-gradient and superconducting RF cavities are absolutely essential in accelerating muon beams quickly enough for interactions to occur before muon decay, and still pose a technological challenge[6].
	\par Additionally, the current transverse emittance[Appendix A] of muon beams produced by conventional methods is too large to produce a reasonable number of muon collisions in a theoretical collider ring. This limits the luminosity of muon beams, which in turn decreases the number of collision events per unit time in the accelerator ring[6][13].
	
	\subsection{Logistical Challenges}
	
	\par An ideal absorber would be composed of hydrogen, which is unique in that it has the lowest atomic number of any element and therefore boasts a large radiation length. Using liquid hydrogen in ionization cooling experiments would reduce multiple (Coulomb) scattering, which contributes to increase in beam emittance. However, storage of liquid hydrogen requires specialised containment vessels, to say nothing of the risk of storing pressurised hydrogen, which introduces the risk of catastrophic combustion on sites responsible for ionization cooling.
	
	
	\section{Two Paths to a Muon Collider}
	
	\subsection{Path 1: Ionization Cooling and MICE}
	
	
	\par The Muon Ionization Cooling Experiment (MICE) is an ongoing research project at the Rutherford Appleton Laboratory in Oxfordshire, England, which seeks to reduce the emittance of muon beams after their production by pion decay. MICE seek to reduce this emittance by passing beams of muons through lithium hydride or liquid hydrogen absorbers, chosen for their low atomic number, $Z$, and hence large radiation lengths[10]. Even in low-$Z$ materials, however, Coulomb scattering still causes “heating” of the beam and increase in beam emittance, which is a significant antagonist to the ionization cooling process. Muon beams, after ionizing atoms within the material, are theoretically “cooled”, i.e. show a reduction in their emittance[10][14]. This is known as six-dimensional cooling[Appendix A][15]. 
	
	
	$$\frac{d\epsilon_{n}}{ds} \approx \frac{-1}{\beta^2}\langle \frac{dE_\mu}{ds} \rangle \frac{\epsilon_{n}}{E_\mu} + \frac{1}{\beta^3}\frac{\beta_{\perp}(0.014GeV)^2}{2E_\mu  m_\mu L_R}$$
	
	\begin{center}
		\par \textit{Muon cooling equation, sourced from M. Uchida's presentation on MICE at RuPAC, 2016[18]. $\frac{d\epsilon_{n}}{ds}$ is rate of change of normalised-emittance with respect to distance travelled through the absorber. $\beta$ is muon velocity as a fraction of the speed of light, $E_\mu$ and $m_\mu$ are muon energy and mass, $\beta_{\perp}$ is the lattice betatron function at absorber, $L_R$ is absorber radiation length.}
	\end{center}
	
	
	
	\par Measurements taken at MICE strongly indicate the success of ionization cooling in reducing muon beam emittance, with the greatest effects seen on high amplitude beams with large transverse emittance[10]. The reduction in longitudinal momentum as a result of the cooling process can be combated via acceleration in radiofrequency cavities[6]. However, the reported 6$\%$ emittance reduction still produces muon beams two orders of magnitude above that required for collider operation[7].
	
	
	
	\begin{center}
		\includegraphics[scale=0.6]{upstream_downstream_nature.png}
		\par \textit{Data collected from the teams at MICE[10]. Upstream and downstream muon distributions are shown for four absorbers- an empty and full liquid hydrogen vessel, no absorber and a lithium hydride absorber.}
		\par 4-140, 6-140, 10-140 \textit{denotes beams with} 140 MeV \textit{momenta with normalised r.m.s. upstream emittances of} 4mm, 6mm \textit{and} 10mm \textit{respectively}.
	\end{center}	
	
	
	
	
	
	\subsection{Path 2: LEMMA}[8][16]
	
	\par A highly attractive alternative to the use of proton-based production of high emittance muon beams is the theoretical LEMMA, which utilises a positron source in the production of muon beams whose low emittance eliminates the necessity of ionization cooling. Such a method would involve positron beam collisions with a fixed target, producing muon-antimuon pairs after positron annihilation with target electrons. However, efficiency from muon productions using such a process is severely limited.
	
	$$eff(\mu^+ \mu^-)=\frac{n(\mu^+ \mu^-)}{N_b(e^+)}$$
	
	\begin{center}
		\par \textit{From M. Boscolo's paper[8] showing the expression for muon conversion efficiency. $N_b(e^+)$ is no. of positrons per "bunch", $n(\mu^+ \mu^-)$ is no. of muon-antimuon pairs produced per positron bunch.}
	\end{center}
	
	
	
	\par Some simulations conducted calculate a muon conversion efficiency of just $10^{-7}$[8]. This would necessitate a design which allowed several passes of the positron beam through a positron ring, allowing multiple muon-antimuon production events and thus counteracting the incredibly low efficiency of muon conversion.
	
	\section{Muon Colliders- Future Potential}
	
	\subsection{Use in Neutrino Factories}
	\par Muon colliders also hold the potential for the creation of highly pure, precisely characterized beams of neutrinos[6][7]. Much of the technology required for the construction and operation of a muon collider is also essential in the creation of muons and antimuons, which primarily decay via the following processes:
	
	$$\mu^+\rightarrow e^+ \nu_e \bar{\nu}_\mu$$
	$$\mu^-\rightarrow e^- \bar{\nu}_e \nu_\mu$$
	
	\begin{center}
		\par \textit{Typical muon and antimuon decay pathways, showing the viability of using muons and antimuons in the production of neutrinos.}
	\end{center}
	
	\par These processes supply high-energy physicists with precision sources of neutrinos, which could serve as sources for a future Neutrino Factory[8]. Thus, much of the technology employed with the aim of producing, accelerating and storing muons would be invaluable in the creation of a future Neutrino Factory.
	
	\par In particular, the nuSTORM experiment would benefit significantly from the existence of an operational muon collider, or even just a muon storage ring. The experiment, which seeks to use neutrino beams characterised to a high level of precision, could lead to a greater understanding of nuclear physics through use of a neutrino probe, which is sensitive to isospin and flavour[17].
	
	\subsection{How Far Away is a Muon Collider?}
	\par The future of muon colliders is anything but certain. Estimates for the cost of a full conceptual design report (CDR) do not currently exist, though the 2022 CERN Yellowpaper on muon colliders provides a tenative estimate that, circa 2044, a muon collider could be operational. Current research and development plans focus on the construction of colliders at the 3 TeV and 10 TeV energy levels, and it has been proposed that magnets similar to those constructed for the HL-LHC could be used in the construction of such muon colliders.[18] 
	
	
	\section{Conclusions}
	
	\par Muon colliders are an exciting, highly promising and, as of yet (April 2023) unrealised technology. Significant advantages are provided by the construction of muon colliders over their hadron counterparts; muons are fundamental, point-like particles and can contribute their energy completely to a collision, whereas hadron collisions occur between constituent particles (such as quarks) which only carry a fraction of the hadron energy. This allows muon beams of a given energy to probe far higher energy levels than hadron beams of comparable energy, opening the possibility of producing multiple Higgs events and investigating the coupling of the Higgs field to itself. Additionally, these higher energy levels may yield even more fruitful results, such as the discovery of entirely new particles, such as those predicted by supersymmetry, furthering our understanding of the Standard Model. Disproving the existence of these particles would be equally valuable in furthering our progress. Muon beams also suffer less radiative energy losses when passing around a collider ring than electron beams owing to their greater mass- this amounts to a reduction on the order of $10^9$. Developing the requisite technology for muon colliders also opens the door for construction of precise neutrino beams for use in Neutrino Factories, which would be incredibly valuable assets in the study of neutrinos going forward.
	\par The current path to building a muon collider is fraught with difficulty; beams of muons produced by conventional methods (proton beams colliding with a target) have too high of an emittance to serve as viable candidates for collisions in an accelerator ring. Additionally, there are technological impediments to consider: incredibly high-powered RF cavities with large field gradients are required to accelerate muons to a sufficient velocity before they decay- the muon rest-frame lifetime is incredibly short at 2.2$\mu$s. 
	\par Current methods to address the emittance problem include ionization cooling experiments, such as the one at MICE in England, or production of lower-emittance muon beams using a positron source, employing a method such as the one proposed in the construction of a LEMMA-type accelerator design. However, MICE have only observed emittance reductions of around 6$\%$, far less than what would be required to produce useable beams of muons.
	\par In the opinion of the author, muon colliders will one day be a reality, overtaking proposed colliders such as the FCC, owing to their high efficacy and use of fundamental particles over hadrons.
	
	
	\section{Acknowledgements}
	\par This research review would have posed an insurmountable task had it not been for Dr Uchida’s guidance and tutelage in the months leading to its submission. Not only did Dr Uchida introduce me to the concept of muon colliders, but she was also responsible for teaching me a large volume of the prerequisite background information necessary for this review to be (somewhat) informed. The author also thanks Prof Potter for attending an early presentation of this research review and advising me to go easy on the political angle originally present in the review.
	
	
	\section{Appendices}
	
	\subsection{Appendix A: Emittance}[19][20]
	\par Emittance describes the cross-section of particle beams in phase-space, consisting of six dimensions: three spatial dimensions and three dimensions in reciprocal (momentum) space. Currently, the high emittance of muon beams produced by pion decay reduces the viability of muon colliders, as too few collision events would be observed following muon injection into the ring.
	\par Reducing the emittance of muon beams currently produced by collision of proton beams with a fixed target is one of the foremost challenges obstructing the creation of a muon collider. Beams with large emittance volumes in phase space are less likely to undergo collision within a collider ring and as such are of limited value to high-energy physicists.
	\par Measuring and controlling particle emittance requires multiple points of measurement along the beam path; gleaning the nature of a particle’s momentum from a single point of measurement is not possible.[17]
	
	$$\epsilon_{n} = \frac{1}{m_\mu}\sqrt[4]{\textrm{det}\Sigma}$$
	\begin{center}
		\par \textit{Expression for normalised emittance.}
	\end{center}
	
	
	$$\textrm{Sigma is the covariance matrix for $6$D phase space:}$$
	
	$$\left( \begin{array}{cccccc}
		\sigma_{xx}^2 & \sigma_{xy}^2 & \sigma_{xz}^2 & \sigma_{xp_x}^2 & \sigma_{xp_y}^2 & \sigma_{xp_z}^2 \\
		\sigma_{yx}^2 & \sigma_{yy}^2 & \sigma_{yz}^2 & \sigma_{yp_x}^2 & \sigma_{yp_y}^2 & \sigma_{yp_z}^2 \\
		\sigma_{zx}^2 & \sigma_{zy}^2 & \sigma_{zz}^2 & \sigma_{zp_x}^2 & \sigma_{zp_y}^2 & \sigma_{zp_z}^2 \\
		\sigma_{p_xx}^2 & \sigma_{p_xy}^2 & \sigma_{p_xz}^2 & \sigma_{p_xp_x}^2 & \sigma_{p_xp_y}^2 & \sigma_{p_xp_z}^2 \\
		\sigma_{p_yx}^2 & \sigma_{p_yy}^2 & \sigma_{p_yz}^2 & \sigma_{p_yp_x}^2 & \sigma_{p_yp_y}^2 & \sigma_{p_yp_z}^2 \\
		\sigma_{p_zx}^2 & \sigma_{p_zy}^2 & \sigma_{p_zz}^2 & \sigma_{p_zp_x}^2 & \sigma_{p_zp_y}^2 & \sigma_{p_zp_z}^2 \\
	\end{array} \right)\ $$
	
	$$\sigma_{xy}^2 = \overline{xy} -\bar{x}\bar{y}$$
	
	
	
	\subsection{Appendix B: Muon Production}
	\par Muons can be produced by striking a target with a proton beam, producing secondary beams of pions, kaons, and, crucially, muons[10]. Another possibility, exploited by theoretical LEMMA designs, involves production of muon beams from a positron source, which produces low-emittance beams with low efficiency.
	
	
	\subsection{Appendix C: Luminosity}[13]
	\par The parameter of “luminosity” is often used to characterise the quality and performance of collisions in a particle collider. Roughly speaking, it is the constant of proportionality relating the collision cross section and rate of collision events:
	
	$$\frac{dR}{dt} = \mathcal{L} * \sigma_p$$
	
	\par Giving luminosity dimensions of $L^{-2}T^{-1}$. Luminosity is proportional to the target’s density ($\rho$) and length ($l$), and proportional to the flux ($\phi$) of the incident beam in the collision for fixed-target collisions:
	
	$$\mathcal{L}_{fixed target} = \phi \rho l$$
	
	\par And, more generally, for collisions where we have two incident beams colliding with each other (hence a situation where there is no single “target” and density is not fixed in time):
	$$\mathcal{L} \propto  \int\int\int\int_{-\infty}^{+\infty} \rho_{beam1}(x,y,s,-s_0)\rho_{beam2}(x,y,s,s_0) \,dxdydsds_0 $$
	
	\par Particle beams with insufficient luminosity are unlikely to be useful in creating a particle collider, as beams with low luminosity provide fewer collision events and thus reduce the amount of useful information which can be gleaned from any given collision.
	
	
	\subsection{Glossary}
	\par \textit{Radiation length}- Quantifies how much energy an electromagnetic particle loses when passing through a material. Materials with small radiation lengths have the greatest dissipative effect on incident particle energies per unit length.
	
	\section{References}
	[1] home.web.cern.ch/science/physics/higgs-boson/how (CERN), \textit{How Did We Discover the Higgs Boson?}\newline
	[2] home.web.cern.ch/scienceaccelerators/large-electron-positron-collider (CERN), \textit{The Large Electron-Positron Collider}.\newline
	[3] home.web.cern.ch/scienceaccelerators/intersecting-storage-rings (Cern), \textit{The Intersecting Storage Rings}.\newline
	[4] K. Johnsen, \textit{CERN Intersecting Storage Rings} (1973).\newline
	[5] R. Aymar, \textit{The Origin of LEP and LHC} (2014).\newline
	[6] K. Long, D. Lucchesi et al., \textit{Muon Colliders: Opening New Horizons for Particle Physics} (2020).\newline
	[7] The Muon Collider Working Group (Jean Pierre Delahaye, Marcella Diemoz et al.), \textit{Input to the European Particle Physics Strategy Update: Muon Colliders}.\newline
	[8] M. Boscolo et al., \textit{The Future Prospects of Muon Colliders and Neutrino Factories}.\newline
	[9] C. Potter et al., \textit{Particle and Nuclear Physics Lecture Handout, 2022-23, Cavendish Laboratory, Department of Physics, University of Cambridge}.\newline
	[10] Nature Vol 578, \textit{Demonstration of cooling by the Muon Ionization Cooling Experiment} (2020).\newline
	[11] A. Moretti, N. Holtkamp et al., \textit{RF Cavities for the Muon and Neutrino Factory Collaboration Study}, XX International Linac Conference, Monterey, California.\newline
	[12] K. D. French, \textit{Development of a dielectric loaded RF cavity for a muon accelerator} (2010).\newline
	[13] W. Herr \& B. Muratori, \textit{Concept of luminosity}.\newline
	[14] Science and Technology Facilities Council (STFC)- \textit{MICE - Muon Ionization Cooling Experiment}.\newline
	[15] Chun-xi Wang \& Kwang-Je Kim, \textit{Linear Theory of 6D Ionization Cooling}.\newline
	[16] M. Antonelli \& P. Raimondi, \textit{Snowmass Report: Ideas for Muon Production from Positron Beam Interaction on a Plasma Target} (2013).\newline
	[17] K. Long 2018 J. Phys.: \textit{The nuSTORM Experiment}, Conf. Ser. 1056 012033\newline
	[18] FERMILAB-PUB-22-338-AD, CERN Yellow Reports: Monographs, CERN-2022-001 \newline
	[19] Barletta, Spentzouris \& Harms, \textit{USPAS notes}, US Particle Accelerator School, Fermi National Accelerator Laboratory.\newline
	[20] M. Uchida, \textit{The MICE Muon Ionization Cooling Experiment: Progress and First Results}, RuPac (2016).
	
	
\end{document}
